% !Mode:: "TeX:UTF-8"

\documentclass[12pt,oneside]{book}

\usepackage{mybook} 
\usepackage{mybookcover}


\title{时评札记}
\author{Wander}
\hypersetup{
    pdftitle={时评札记},
    pdfauthor={Wander},
    pdfcreator={Wander},
    pdfsubject={新闻},
}
  

\begin{document}
\makemytitle

\flypage{感谢上帝}


\frontmatter 
\addchtoc{前言}
\chapter*{前言}
我关于当下时事的一些评论。写在此处,图个自由自在。



\addchtoc{目录}
\setcounter{tocdepth}{2}    
\tableofcontents


\mainmatter


\part{2025年}
\chapter{九三阅兵习近平向普京谈寿命}
九三阅兵之时,习近平通过俄文翻译向普京说:“过去,超过70岁的人很少见;如今人们说,70岁还只是个孩子。”

后来习近平又说道:“有预测呢,本世纪呢,可能可以活到150岁。”

是必有佞臣谄笑对习近平说起这些,习近平才有感而发。

此佞臣高官对治下人民则又是另外一番嘴脸了:“三十五岁,半截身子都进土了。”

我评曰:习近平一百五十岁,高官一百岁,平民七十岁,那些个贱民三十五岁怎么还不去死啊。


\chapter{无国之人应该拒绝当兵杀人}
如果一个人因为资财或者其他限制条件,使得他不得不局限于一国之内,那么这个人可称之为半个无国之人,因为他连主动选择的可能性都没有,那么称他对于本国有十分的忠诚是很值得怀疑的。

再继而那半个无国之人在本国之内并无半点的政治权利,好像游离于某个系统之外,这样的人就可以称得上是完全的无国之人了。

当兵杀人本是不义之举,有了杀孽是要遭到上天的报应和惩罚的,有国之人权衡利弊,愿为国之利器,实在是觉得此国确为自己所有,确实有莫大的利益在其中。

至于无国之人,则完全没有此类权衡利弊得失的麻烦,应果断拒绝当兵杀人。

我认为那些动刀兵的国家,从战略上,应该尽早确立宣传上的攻势和实践中的政治策略,将那些无国之人的群体顺利过渡到本国阵营之内,而不要逼反了他们,不求他们做出多大的贡献,只这顺利过渡一条,已足见天地人心。



\chapter{再谈海瑞}
最近在看《大明王朝1566》,对海瑞的形象有些好奇,于是逮着豆包问了几个问题,问着问着豆包就开始吞问答无法进行下去了,当然,这里不是批评豆包的意思,国内的大语言模型应该都一个样。这也是我在这里开设本项目的原因之一。

其实都已经是一些陈谷子的事情了,共产党早就是一个人尽皆知的婊子了,却还在那里装作守身如玉的样子,实在可笑。

海瑞的形象一直都是正面的积极的,毛泽东在文革之前也是肯定赞扬的态度,怎么到了文革就成了一个被批判的对象,之前的问题豆包还避重就轻地就文革的一些事情做了说明。我大体了解了一下,然后得出结论,所谓的对《海瑞罢官》的批判,根本和海瑞没有半毛钱关系,而是借着海瑞直臣之名,来打压彭德怀。所以后面我问豆包,可不可以这样认为,文革对海瑞形象的批判,和历史上的海瑞形象无关,本质上就是捕风捉影,来行党内斗争之实。豆包不答。

\begin{enumerate}
\item 1965年11月10日,毛泽东正式批准发表姚文元的《评新编历史剧〈海瑞罢官〉》。并且直接了当地指出彭德怀也是海瑞。
\item 1966年6月,彭德怀被批判。
\item 1966年12月,彭德怀被强行绑架之后押回北京。
\end{enumerate}

什么江青林彪四人帮,都不过是毛泽东的手下,文革是毛泽东发动和主导的,这是事实,什么皇帝还是圣明的,都是底下臣子在坏事的幼稚愚民思想,不值一提。

此处我不想深谈文革,也不想谈什么主义和国际大势,只是提醒大家一点,一个很明白的事实,那就是从大跃进到庐山会议,到文革到彭德怀被批斗致死,这一系列事件都是一脉相承的。


\end{document}


