% !Mode:: "TeX:UTF-8"

\documentclass[12pt,oneside]{book}

\usepackage{mybook} 
\usepackage{mybookcover}


\title{时评札记}
\author{Wander}
\hypersetup{
    pdftitle={时评札记},
    pdfauthor={Wander},
    pdfcreator={Wander},
    pdfsubject={新闻},
}
  

\begin{document}
\makemytitle

\flypage[40pt][40pt]{你要以\\全部的心知\\全部的灵魂\\全部的头脑\\以及\\全部的气力\\来爱主你的神\\}


\frontmatter 
\addchtoc{前言}
\chapter*{前言}
我关于当下时事的一些评论。写在此处,图个自由自在。



\addchtoc{目录}
\setcounter{tocdepth}{2}    
\tableofcontents


\mainmatter


\part{2025年}
\chapter{关于基督教的我国态度}
时不时的有一些想向上表忠心的红小鬼发表一些逆天的宗教言论,唯恐天下不乱。下面阐述几个关于基督教的关键要点。

\begin{itemize}
\item 基督教在我国人民最困难的时候,是有大功劳的。这一困难时期不局限在抗日时期,而是一直可以向上推很远。有的时候闹饥荒,不是一个人,一家人,而是一村人一村人的被救命。中国有句老话叫做救命之恩即是再生父母,对于那些人以父母之义敬奉基督教,恰恰是中华民族的好儿女,而那些指责他们背叛的人反而是不知礼,中华自古皆有此理,执我华夏礼仪者不论出身皆我华夏儿女,更何况他们本就是中华儿女,而那些攻击他们的人,实不知礼而背弃祖宗文化者是也。凡事讲究一个道理,如果他们忘了那份救命之恩,那谁能指望他们对民族忠义,对国家忠诚呢。
\item 少给国家惹事。国家财政,国家公务人员,国家公权力,都各有各的职能,都各有各的重要的事情要做,那些无事生非的人,没事尽找麻烦的人,才是真的是流氓祸害之流啊。宪法规定宗教自由,各有各的理解,你就是理解力再低也能看出那个意思,那就是少给国家惹事,大家相安无事最好。对于那些无事生非之徒,大家是碍于脸面不能责怪你,我若是官老爷你看我不狠狠地打你几大板子。
\item 基督教的统战价值是永远不能碰的红线。不战而屈人之兵,善之善者也。虽然中国大陆地理位置优厚,但谁也不能保证中国将来不会跟一个基督教信众占多数的国家发生了战争,而如果中国对待基督教在外面的名声不错,那么很多基督徒的参战意愿就会降低,不指望他们能够倒戈卸甲,就是让他们不愿意参战这一条,就构成了我国必须保持基督教的统战价值这条红线不能碰。兵者,国之大事也,当善始谋之。
\end{itemize}






\chapter{最近的两件小事}
最近的两件小事,主流媒体都漠然视之。

美国的乌克兰移民伊琳娜遭遇不幸,被杀死之后,特朗普政府连基本的帮助她父亲收尸这样的举手之劳都做不到,在柯克那些宣扬仇恨的言论鼓动下,很多美国网民更是调侃、幸灾乐祸。而搅屎棍柯克死了之后,人们因为不喜欢他的政治观点,多说了两句就被风闻奏事,各种不合理的政治打压。现在的美国政府高层不是试图解决问题,更是煽风点火。如上种种都让天佑美利坚这个词成了笑话。MAGA的那群人忘了一个最根本的问题,那就是他们现在是执政党,如果有问题,那么解决问题是他们的责任,解决不好问题是他们的问题,而不是搞内部政治斗争,宣扬仇恨,制造分裂。

成都玉石公园一人上吊自杀了,现场无打斗痕迹,仅留有一张手写纸条:“别害怕,我是自杀。我真骑不动了,本来想找人少、树不错、风景又好的地方,真走不动了。”中国底层人民民生之难实在让人说不尽,中国共产党,你们对人民的压榨能不能轻一些?你们给中国人民制造的那些压力大山,能不能减去一块石头?


\chapter{如何评价查理·柯克其人}
MAGA之心。

在他没有死的那个世界里,MAGA运动最终将给美国带来深重的灾难,甚至是动摇国本,国家内战,美帝彻底崩溃的地步,美国用他自己的特色方式解决了这个潜在问题了。

特朗普失去MAGA之心之后会更加缺少主见和行动意愿,后面会更加顺从身边人的意见。

这就是美国政治的现状,其实这种局面美国开国者早就预见了的,他们根本不信那种大众的盲流\footnote{盲从之流。}的民主,在制度上各种防范。

这个科克对美国现阶段那些盲流青少年号召力极强,特朗普还不冷处理,小心惹火上身。



\chapter{评柯克被刺杀事件}
美国当下有很有问题,特朗普政府当前的政策更多地是照顾那些不得不找工作的底层人士,为此禁移民,促进本土产业等等,这些都是共和党的基本盘,特朗普照顾得很好。但特朗普政府现在的政策是完全抛弃了一些人群:那就是大学就业人群。

这类人群在当下的人工智能冲击下前景非常迷茫,特朗普政府不仅不从政策上帮助他们,有时还会从言语上讥讽他们。总的来说这类人群所占人口基数是偏小的,而且他们的问题要解决起来会更加的棘手,并且他们似乎相对来说经济问题应该会更小一些,也觉得他们在政治会更加温和一些,加上他们大多是民主党那边的,所以特朗普政府干脆完全舍弃掉他们了。

柯克被刺杀事件不会改变美国的本质,恰恰反映了美国的本质。柯克被刺杀事件可能并不是某个重要的标志性事件,因为我不会觉得特朗普会因此改变现有的施政方针。柯克被刺杀事件并不是一个媒体夸大其词的美式政治暴力升级信号,只是当下美国政治被彻底漠视的一小部分人群,面对一个还在不停嘲弄他们的人,因愤怒而产生的激进举动。



\chapter{评九三阅兵坚毅女兵上的满脸汗水}
十几年前我在成都漫游,也是见到了同样一个如此坚强的女性,大约二三十的年纪,背上扛着约有半个身子高的塑料盘堆,顿了顿,稳了稳,然后坚毅地走了过去。这样的画面我至今还记得。

近年来兴起的什么女权,我向来不感兴趣,但我对于类似上面这样的女性,一直都是心怀敬畏之心的。并总觉得,她们值得更好的命运。




\chapter{九三阅兵习近平向普京谈寿命}
九三阅兵之时,习近平通过俄文翻译向普京说:“过去,超过70岁的人很少见;如今人们说,70岁还只是个孩子。”

后来习近平又说道:“有预测呢,本世纪呢,可能可以活到150岁。”

是必有佞臣谄笑对习近平说起这些,习近平才有感而发。

此佞臣高官对治下人民则又是另外一番嘴脸了:“三十五岁,半截身子都进土了。”

我评曰:习近平一百五十岁,高官一百岁,平民七十岁,那些个贱民三十五岁怎么还不去死啊。


\chapter{无国之人应该拒绝当兵杀人}
如果一个人因为资财或者其他限制条件,使得他不得不局限于一国之内,那么这个人可称之为半个无国之人,因为他连主动选择的可能性都没有,那么称他对于本国有十分的忠诚是很值得怀疑的。

再继而那半个无国之人在本国之内并无半点的政治权利,好像游离于某个系统之外,这样的人就可以称得上是完全的无国之人了。

当兵杀人本是不义之举,有了杀孽是要遭到上天的报应和惩罚的,有国之人权衡利弊,愿为国之利器,实在是觉得此国确为自己所有,确实有莫大的利益在其中。

至于无国之人,则完全没有此类权衡利弊得失的麻烦,应果断拒绝当兵杀人。

我认为那些动刀兵的国家,从战略上,应该尽早确立宣传上的攻势和实践中的政治策略,将那些无国之人的群体顺利过渡到本国阵营之内,而不要逼反了他们,不求他们做出多大的贡献,只这顺利过渡一条,已足见天地人心。



\chapter{再谈海瑞}
最近在看《大明王朝1566》,对海瑞的形象有些好奇,于是逮着豆包问了几个问题,问着问着豆包就开始吞问答无法进行下去了,当然,这里不是批评豆包的意思,国内的大语言模型应该都一个样。这也是我在这里开设本项目的原因之一。

其实都已经是一些陈谷子的事情了,共产党早就是一个人尽皆知的婊子了,却还在那里装作守身如玉的样子,实在可笑。

海瑞的形象一直都是正面的积极的,毛泽东在文革之前也是肯定赞扬的态度,怎么到了文革就成了一个被批判的对象,之前的问题豆包还避重就轻地就文革的一些事情做了说明。我大体了解了一下,然后得出结论,所谓的对《海瑞罢官》的批判,根本和海瑞没有半毛钱关系,而是借着海瑞直臣之名,来打压彭德怀。所以后面我问豆包,可不可以这样认为,文革对海瑞形象的批判,和历史上的海瑞形象无关,本质上就是捕风捉影,来行党内斗争之实。豆包不答。

\begin{enumerate}
\item 1965年11月10日,毛泽东正式批准发表姚文元的《评新编历史剧〈海瑞罢官〉》。并且直接了当地指出彭德怀也是海瑞。
\item 1966年6月,彭德怀被批判。
\item 1966年12月,彭德怀被强行绑架之后押回北京。
\end{enumerate}

什么江青林彪四人帮,都不过是毛泽东的手下,文革是毛泽东发动和主导的,这是事实,什么皇帝还是圣明的,都是底下臣子在坏事的幼稚愚民思想,不值一提。

此处我不想深谈文革,也不想谈什么主义和国际大势,只是提醒大家一点,一个很明白的事实,那就是从大跃进到庐山会议,到文革到彭德怀被批斗致死,这一系列事件都是一脉相承的。


\end{document}


